
\chapter{无外场、空间均匀的线性波动问题}

\section{静电波解 \label{section:321}}

$\vector{B}_1=0,\vector{E}_1=\grad{\phi_1} $

线性化弗拉索夫方程
\begin{equation}
\pdv{f_{\alpha,1}}{t} + \vector{v} \cdot \pdv{f_{\alpha,1}}{\vector{r}}
- \frac{q_\alpha}{m_\alpha} \pdv{\phi_1}{\vector{r}} \cdot \pdv{f_{\alpha,0}}{\vector{v}}
= 0
\end{equation}

线性化泊松方程
\begin{equation}
\laplace{\phi_1} = - \frac{1}{\varepsilon_0} \sum_\alpha q_\alpha \int f_{\alpha,1} \d^3 \vector{v}
\end{equation}

傅里叶变换
\begin{subequations}\begin{gather}
-\im\omega f_{\alpha,k} - \left.f_{\alpha,k}\right|_{t=0} + \im \vector{k} \cdot \vector{v} f_{\alpha,k}
- \im \frac{q_\alpha}{m_\alpha} \phi_k \vector{k} \cdot \pdv{f_{\alpha,0}}{\vector{v}}
= 0 \\
k^2 \phi_k = \frac{1}{\varepsilon_0} \sum_\alpha q_\alpha \int f_{\alpha,k} \d^3 \vector{v}
\end{gather}\end{subequations}

化简得
\begin{subequations}\begin{align}
f_{\alpha,k} &= - \frac{q_\alpha}{m_\alpha} \frac{\phi_k}{\omega - \vector{k} \cdot \vector{v}} \vector{k} \cdot \pdv{f_{\alpha,0}}{\vector{v}}
+ \im \frac{f_{\alpha,k}(0)}{\omega - \vector{k} \cdot \vector{v}} \label{eq:321temp1} \\
\phi_k &= \frac{1}{\varepsilon_0 k^2} \sum_\alpha q_\alpha \int f_{\alpha,k} \d^3 \vector{v} \label{eq:321temp2}
\end{align}\end{subequations}

将 \ref{eq:321temp1} 代入 \ref{eq:321temp2} 得:
\begin{subequations}\begin{align}
\varepsilon(\omega, \vector{k}) &= 1 + \sum_\alpha \chi_\alpha(\omega, \vector{k}) \label{eq:321varepsilon}\\
\chi_\alpha(\omega, \vector{k}) &= \frac{\omega_{p,\alpha}^2}{k^2}
\int \frac{\vector{k} \cdot \pdv{}{\vector{v}} \hat{f}_{\alpha,0}}{\omega - \vector{k} \cdot \vector{v}} \d^3 \vector{v} \label{eq:321chi}\\
\phi_k &= \frac{\im}{k^2 \varepsilon_0 \varepsilon(\omega, \vector{k})} \sum_\alpha q_\alpha \int \frac{f_{\alpha,k}(0)}{\omega - \vector{k} \cdot \vector{v}} \d^3 \vector{v}
\end{align}\end{subequations}
其中,$\varepsilon$ 为相对介电函数,$\chi_\alpha$ 为极化率。

对于本征值问题,不考虑时间起点,即 $f_{\alpha,k}(0) = 0$,为了使电场扰动不为零,则必须有
\begin{equation}
\varepsilon(\omega, \vector{k}) = 0
\end{equation}
这样就可以得出色散关系。

对于初值问题,则必须考虑时间起点 $f_{\alpha,k}(0)$ 的影响。

\subsection{束等离子体的本征模}

粒子束:
\begin{equation}
f_{\alpha,0} = n_{\alpha,0} \delta(\vector{v} - \vector{u}_\alpha)
\end{equation}
极化率
\begin{equation}\begin{aligned}
\chi_\alpha(\omega, \vector{k}) &= \frac{\omega_{p,\alpha}^2}{k^2} \int \left[
\pdv{}{\vector{v}} \cdot \left( \frac{\vector{k} \hat{f}_{\alpha,0}}{\omega - \vector{k} \cdot \vector{v}} \right)
- \hat{f}_{\alpha,0} \vector{k} \cdot \pdv{}{\vector{v}}\left( \frac{1}{\omega - \vector{k} \cdot \vector{v}} \right)
\right] \d^3 \vector{v} \\
&= - \frac{\omega_{p,\alpha}^2}{k^2} \int
\hat{f}_{\alpha,0} \vector{k} \cdot \frac{\vector{k} \cdot \vector{I}}{\left( \omega - \vector{k} \cdot \vector{v} \right)^2}
\d^3 \vector{v} \\
&= - \frac{\omega_{p,\alpha}^2}{\left( \omega - \vector{k} \cdot \vector{u}_\alpha \right)^2}
\end{aligned}\end{equation}
相对介电函数
\begin{equation}
\varepsilon = 1 - \sum_\alpha \frac{\omega_{p,\alpha}^2}{\left( \omega - \vector{k} \cdot \vector{u}_\alpha \right)^2}
\end{equation}

假设离子静止 $\vector{u}_i = \vector{0}$,电子流速 $\vector{u}_e \parallel \vector{k}$,则色散关系为
\begin{equation}
\frac{\omega_{p,e}^2}{\left( \omega - k u_e \right)^2} + \frac{\omega_{p,i}^2}{\omega^2} = 1
\end{equation}

\subsubsection{忽略离子响应}

在忽略离子响应($\omega_{p,e} \gg \omega_{p,i}$)的情况下,色散关系可以简化为
\begin{equation}
\omega = k u_e \pm \omega_{p,e}
\end{equation}
得到快/慢波模。

\paragraph{波能}
\begin{equation}
\pdv{\left(\omega \varepsilon\right)}{\omega}
= \omega \pdv{\varepsilon}{\omega}
= \pm \frac{2 \omega}{\omega_{p,e}}
\end{equation}
则快波为正能波,慢波为负能波。

\paragraph{扰动密度}
\begin{equation}
n_{e,k} = \int f_{e,k} \d^3\vector{v}
= - \frac{\varepsilon_0}{e} k \phi_k
\end{equation}
快/慢波模同相。

\paragraph{平均扰动速度}
考虑粒子流的扰动
\begin{equation}
\vector{\Gamma}_{\alpha,1} = \int \vector{v} f_{\alpha,1} \d^3\vector{v}
= n_{\alpha,0} \vector{u}_{\alpha,1} + n_{\alpha,1} \vector{u}_{\alpha,0}
\end{equation}
则平均扰动速度为
\begin{equation}\begin{aligned}
u_{e,k} &= \frac{n_{e,k}}{n_{e,0}} \left(\frac{\omega}{k} - u_e \right) \\
&= \mp \sqrt{\frac{\varepsilon_0}{m_e n_{e,0}}} k \phi_k
\end{aligned}\end{equation}
快/慢波模反相。

\paragraph{平均能量密度}

\subsection{束等离子体波模分析}

介电函数
\begin{equation}
\varepsilon = 1 - \sum_\alpha \frac{\omega_{p,\alpha}^2}{\left( \omega - \vector{k} \cdot \vector{u}_\alpha \right)^2}
\end{equation}
不显含虚部,称为 \textbf{反应型},不稳定性/阻尼来自于波-波相互作用。与之对应的,介电函数显含虚部的称为 \textbf{耗散型},其不稳定性/阻尼的来源还包括波-粒子相互作用。

\paragraph{典型的束-等离子体系统}
离子不响应,本地冷电子密度 $n_{e,0}$ 远大于电子束密度 $n_{b,0}$。色散关系:
\begin{equation}
\frac{\omega_{p,b}^2}{\left( \omega - k u_b \right)^2} + \frac{\omega_{p,e}^2}{\omega^2} = 1
\end{equation}

极限情况:
\begin{itemize}
    \item 当不存在电子束时,波模退化为简单的电子等离子体波:$\omega = \omega_{p,e}$。
    \item 当不存在本底电子时,波模退化为上节所述的快/慢波:$\omega = k u_b \pm \omega_{p,b}$。
    \item 两者都存在时,会产生波模耦合:
\begin{equation}
\left(\omega - \omega_1\right) \left(\omega - \omega_1\right) = \epsilon
\end{equation}
其中,$\epsilon$ 为耦合系数。若 $\epsilon < - (\omega_1 - \omega_2)^2 / 4$,则 $\omega_i > 0$,存在不稳定性。当共振($\omega_1 = \omega_2$)时,$\omega_i = \sqrt{-4\epsilon}$ 最大。
\end{itemize}

渐进行为:
\begin{enumerate}
    \item 当 $\omega \gg k u_b$ 时,色散关系
\begin{equation}\begin{gathered}
\frac{\omega_{p,b}^2}{\omega^2} + \frac{\omega_{p,e}^2}{\omega^2} = 1 \\
\omega^2 = \omega_{p,e}^2 + \omega_{p,b}^2 \approx \omega_{p,e}^2
\end{gathered}\end{equation}
    \item 当 $\omega \ll k u_b$ 时,色散关系
\begin{equation}\begin{gathered}
\frac{\omega_{p,b}^2}{k^2 u_b^2} + \frac{\omega_{p,e}^2}{\omega^2} = 1 \\
\omega^2 = \omega_{p,e}^2 \left/\left(1 - \frac{\omega_{p,b}^2}{k^2 u_b^2} \right)\right.
\end{gathered}\end{equation}
    \item 当 $\omega \sim k u_b \gg \omega_{p,e}$ 时,色散关系
\begin{equation}\begin{gathered}
\frac{\omega_{p,b}^2}{\left(\omega - k u_b\right)^2} = 1 \\
\omega = k u_b \pm \omega_{p,b}
\end{gathered}\end{equation}
    \item 当 $\omega \sim k u_b \ll \omega_{p,e}$ 时,色散关系
\begin{equation}\begin{gathered}
\frac{\omega_{p,b}^2}{\left(\omega - k u_b\right)^2} + \frac{\omega_{p,e}^2}{\omega^2} = 0 \\
\omega = k u_b \left/\left(1 \mp \im \frac{\omega_{p,b}}{\omega_{p,e}} \right)\right.
\approx k u_b \left(1 \pm \im \frac{\omega_{p,b}}{\omega_{p,e}} \right)
\end{gathered}\end{equation}
出现不稳定性:$\dfrac{\omega_i}{\omega_r} = \dfrac{\omega_{p,b}}{\omega_{p,e}} \ll 1$。
    \item 当 $\omega \sim k u_b \sim \omega_{p,e}$ 时,数值求解。
\end{enumerate}

\section{耗散型本征模}

介电函数和色散方程具有如下形式:
\begin{subequations}\begin{align}
\varepsilon &= \varepsilon_\re + \im \varepsilon_\im, \quad \varepsilon_\im \neq 0 \\
\omega &= \omega_\re + \im \omega_\im, \quad \omega_\im \neq 0
\end{align}\end{subequations}

\subsection{
    弱耗散
    \texorpdfstring{($\varepsilon_\im \ll \varepsilon_\re$、$\omega_\im \ll \omega_\re$)}{}
    情况下的色散方程
}

对介电函数进行 Taylor 展开得
\begin{equation}\begin{aligned}
\varepsilon(\omega, \vector{k})
&\approx \varepsilon(\omega_r, \vector{k})
+ \im \omega_i \pdv{\varepsilon}{\omega}(\omega_r, \vector{k}) \\
&\approx \varepsilon_r(\omega_r, \vector{k})
+ \im \left[ \varepsilon_i(\omega_r, \vector{k})
+ \omega_i \pdv{\varepsilon_r}{\omega}(\omega_r, \vector{k}) \right]
\end{aligned}\end{equation}

由 $\varepsilon(\omega, \vector{k}) = 0$,得
\begin{subequations}\begin{align}
\omega_r &= \omega_r(\vector{k}) \\
\omega_i &= - \left.\left(
    \varepsilon_i \left/ \pdv{\varepsilon_r}{\omega} \right.
\right)\right|_{\omega=\omega_r}
\end{align}\end{subequations}

\subsection{耗散项 \texorpdfstring{$\varepsilon_i$}{} 的来源}

考虑 \ref{section:321} 节得到的介电函数:
\begin{subequations}\begin{align}
\varepsilon(\omega, \vector{k}) &= 1 + \sum_\alpha \chi_\alpha(\omega, \vector{k}) \tag{\ref{eq:321varepsilon}}\\
\chi_\alpha(\omega, \vector{k})
&= \frac{\omega_{p,\alpha}^2}{k^2}
\int \frac{\vector{k} \cdot \pdv{}{\vector{v}} \hat{f}_{\alpha,0}}{\omega - \vector{k} \cdot \vector{v}} \d^3 \vector{v} \tag{\ref{eq:321chi}}
\end{align}\end{subequations}
其中,被积函数在 $\vector{k} \cdot \vector{v} = \omega$ 奇异。

将速度分解为垂直波矢方向和平行波矢方向,并利用复平面的路径积分得到
\begin{equation}\begin{aligned}
\chi_\alpha(\omega, \vector{k})
&= \frac{\omega_{p,\alpha}^2}{k^2} \int
    \frac{1}{\omega / k - v_\parallel}
    \ab( \pdv{}{v_\parallel}
        \int \hat{f}_{\alpha,0} \d^2 \vector{v}_\perp
    )
\d v_\parallel \\
&= \frac{\omega_{p,\alpha}^2}{k^2} \PV\int
    \frac{\pdv{}{v_\parallel} \hat{f}_{\alpha,0}}{\omega / k - v_\parallel}
\d^3 \vector{v}
- \im \pi \frac{\omega_{p,\alpha}^2}{k^2}
    \eval{\ab(\pdv{}{v_\parallel}
        \int \hat{f}_{\alpha,0} \d^2 \vector{v}_\perp
    )}_{v_\parallel = \frac{\omega}{k}}
\end{aligned}\end{equation}
则
\begin{subequations}\begin{align}
\varepsilon_r &= 1 - \sum_\alpha \frac{\omega_{p,\alpha}^2}{k^2} \PV \int
    \frac{\pdv{}{v_\parallel} \hat{f}_{\alpha,0}}{\omega / k - v_\parallel}
\d^3 \vector{v} \\
\varepsilon_i &= - \pi \sum_\alpha \frac{\omega_{p,\alpha}^2}{k^2}
    \eval{\ab(\pdv{}{v_\parallel}
        \int \hat{f}_{\alpha,0} \d^2 \vector{v}_\perp
    )}_{v_\parallel = \frac{\omega}{k}}
\end{align}\end{subequations}

相关概念
\begin{description}
    \item[共振条件] $\vector{k} \cdot \vector{v} = \omega$;
    \item[共振粒子] 速度在 $\vector{k} \cdot \vector{v} \approx \omega$ 附近的粒子,与波的相互作用时间长,影响较大;
    \item[非共振粒子] 与波的相互作用时间短,影响较小,但仍有影响;
\end{description}

\subsection{麦克斯韦(Maxwell)分布下的等离子体色散方程}

麦克斯韦(Maxwell)分布:
\begin{equation} \tag{\ref{eq:Maxwell_distribution}} \begin{aligned}
f_{\text{M}, \alpha}
&= n_{\alpha, 0} \left(\frac{m_\alpha}{2 \pi T_\alpha}\right)^{3/2}
\exp{\left[- \frac{m_\alpha}{2 T_\alpha} (\vector{v} - \vector{u}_\alpha)^2\right]} \\
&= \frac{n_{\alpha, 0}}{\sqrt{\pi^3} v_{t, \alpha}^3}
\exp{\left[- \frac{(\vector{v} - \vector{u}_\alpha)^2}{v_{t, \alpha}^2}\right]}
\end{aligned}\end{equation}
则极化率为:
\begin{equation}
\chi_\alpha = \frac{1 + \xi_\alpha Z(\xi_\alpha)}{k^2 \lambda_{D,\alpha}^2},
\quad \xi_\alpha = \frac{\omega}{k v_{t,\alpha}}
\end{equation}
其中,$\lambda_{D, \alpha} = \sqrt{\dfrac{\varepsilon_0 T_\alpha}{q_\alpha^2 n_{\alpha, 0}}}$ 为德拜长度。
\begin{equation}\begin{aligned}
Z(x) &= \frac{1}{\sqrt{\pi}} \int_{-\infty}^{+\infty} \frac{\eu^{- y^2}}{y - x} \d y \\
&= \frac{1}{\sqrt{\pi}} \PV \int_{-\infty}^{+\infty} \frac{\eu^{- y^2}}{y - x} \d y
+ \im \sqrt{\pi} \eu^{-x^2}
\end{aligned}\end{equation}
为等离子体色散函数。

在冷等离子体($\xi_\alpha \gg 1$ 即 $v_{t,\alpha} \ll \omega/k$)近似下,
\begin{subequations}\begin{align}
Z_r(\xi_\alpha) &\approx - \frac{1}{\xi_\alpha} \left(
    1 + \frac{1}{2 \xi_\alpha^2} + \frac{3}{4 \xi_\alpha^4}
\right) \\
\chi_{r, \alpha} &\approx - \frac{\omega_{p,\alpha}^2}{\omega^2} \left(
    1 + \frac{3 k^2 v_{t, \alpha}^2}{2 \omega^2}
\right)
\end{align}\end{subequations}

在热等离子体($\xi_\alpha \ll 1$ 即 $v_{t,\alpha} \gg \omega/k$)近似下,
\begin{subequations}\begin{align}
Z_r(\xi_\alpha) &\approx 2 \xi_\alpha \left(
    \frac{2}{3} \xi_\alpha^2 - 1
\right) \ll 1 \\
\chi_{r, \alpha} &\approx \frac{1}{k^2 \lambda_{D,\alpha}^2}
\end{align}\end{subequations}

\section{平衡分布为麦克斯韦分布情况下的静电本征模}

流体描述:
\begin{itemize}
\item 电子等离子体波(EPW)
\begin{equation} \label{eq:321EPW}
    \omega^2 = \omega_{p,e}^2 + \frac{3}{2} k^2 v_{t,e}^2
    = \omega_{p,e}^2 \left( 1 + 3 k^2 \lambda_{D,e}^2 \right)
\end{equation}
\item 离子声波(IAW)
\begin{equation} \label{eq:321IAW}
    \omega^2 = k^2 \left( \frac{c_s^2}{1 + k^2 \lambda_{D,e}^2} + \frac{3}{2} v_{t,i}^2 \right)
\end{equation}
\end{itemize}

\subsection{电子等离子体波(EPW)}

\begin{itemize}
    \item $\omega / k = v_p \gg v_{t,e} \gg v_{t,i}$
    \item $\omega_{p,e} \gg \omega_{p,i}$
    \item 忽略离子响应,$\chi_i \approx 0$
\end{itemize}

\begin{subequations}\begin{align}
\chi_{\re,e} &\approx - \frac{\omega_{p,e}^2}{\omega^2} \left( 1 + \frac{3 k^2 v_{t,e}^2}{2 \omega^2} \right) \\
\chi_{\im,e} &\approx \frac{\sqrt{\pi}}{k^2 \lambda_{D,e}^2} \xi_e \eu^{- \xi_e^2} \\
\pdv{\chi_{\re,e}}{\omega} &= - 2 \frac{\omega_{p,e}^2}{\omega^3} \left( 1 + \frac{3 k^2 v_{t,e}^2}{\omega^2} \right)
\end{align}\end{subequations}

色散方程
\begin{equation}
\frac{\omega_{p,e}^2}{\omega_\re^2} \left( 1 + \frac{3 k^2 v_{t,e}^2}{2 \omega_\re^2} \right) = 1
\end{equation}
解得
\begin{equation}
\omega_\re^2 = \omega_{p,e}^2 + \frac{3}{2} k^2 v_{t,e}^2
= \omega_{p,e}^2 \left( 1 + 3 k^2 \lambda_{D,e}^2 \right)
\end{equation}

\begin{equation}
\frac{\omega_\im}{\omega_\re} = - \varepsilon_\im \left/ \omega_r \pdv{\varepsilon_\re}{\omega}\right.
= - \sqrt{\frac{\pi}{8}} \frac{1}{k^2 \lambda_{D,e}^2} \exp\left(- \frac{1}{2 k^2 \lambda_{D,e}^2} - \frac{3}{2}\right)
\end{equation}
当 $k \lambda_{D,e} \to 0$ 时,$\omega_i / \omega_\re \to 0$,即长波的阻尼小。
当 $k \lambda_{D,e} = 1 / \sqrt{3}{}$ 时,$\omega_\im / \omega_\re$ 最大,$(\omega_\im / \omega_\re)_{\max} \approx -0.162$,$\omega_\re=\sqrt{2} \omega_{p,e}$。

波能
\begin{equation}
\pdv{\omega \varepsilon_\re}{\omega}
= 2 \frac{\omega_{p,e^2}}{\omega_\re^2}
\approx 2
\end{equation}
场能和动能各占一半。

\subsection{离子声波(IAW)}

\begin{itemize}
    \item $v_{t,e} \gg \omega / k = v_p \gg v_{t,i}$
    \item $\omega_{p,e} \gg \omega_{p,i}$
    \item $\xi_i \gg 1 \gg \xi_e$
\end{itemize}

\begin{subequations}\begin{align}
\chi_{\re,e} &\approx \frac{1}{k^2 \lambda_{D,e}^2} \\
\chi_{\im,e} &= \frac{\sqrt\pi}{k^2 \lambda_{D,e}^2}
    \xi_e \eu^{- \xi_e^2} \\
\chi_{\re,i} &\approx - \frac{\omega_{p,i}^2}{\omega^2}
    \ab[1 + \frac32 \frac{k^2 v_{t,i}^2}{\omega^2}] \\
\chi_{\im,i} &= \frac{\sqrt\pi}{k^2 \lambda_{D,i}^2}
    \xi_i \eu^{- \xi_i^2}
\end{align}\end{subequations}

色散方程
\begin{equation}
\varepsilon_\re
= 1 + \frac{1}{k^2 \lambda_{D,e}^2}
- \frac{\omega_{p,i}^2}{\Re(\omega)^2}
\ab[1 + \frac32 \frac{k^2 v_{t,i}^2}{\Re(\omega)^2}]
= 0
\end{equation}
解得
\begin{subequations}\begin{align}
\omega_\re &= k v_s \\
v_s &= \sqrt{
    \frac{c_s^2}{1 + k^2 \lambda_{D,e}^2} + \frac32 v_{t,i}^2
}
\end{align}\end{subequations}

\begin{subequations}\begin{align}
\varepsilon_\im &=
\frac{\sqrt\pi}{k^2 \lambda_{D,e}^2}
\xi_{\re,e} \eu^{- \xi_{\re,e}^2}
+ \frac{\sqrt\pi}{k^2 \lambda_{D,i}^2}
\xi_{\re,i} \eu^{- \xi_{\re,i}^2} \\
\odv{\varepsilon_\re}{\omega} &=
\frac{2 \omega_{p,i}^2}{\omega_\re^3}
\ab[1 + 3 \frac{k^2 v_{t,i}^2}{\omega_\re^2}] \\
\frac{\omega_\im}{\omega_\re} &=
- \sqrt\frac\pi8 \frac{v_s^3}{c_s^3} \ab[
    \sqrt\frac{m_e}{m_i} \exp\ab(- \frac{v_s^2}{v_{t,e}^2})
    + \ab(\frac{T_e}{\eu T_i})^{3/2}
    \exp\ab(- \frac{T_e}{2 T_i \ab(1 + k^2 \lambda_{D,e}^2)})
]
\end{align}\end{subequations}

\begin{itemize}
    \item 长波,且 $T_e \gg T_i$ 时,有 $v_s = \omega / k \gg v_{t,i}$、
    $\ab|\omega_\im| \ll \omega_\re$;
    \item 短波时,有 $v_s \sim v_{t,i}$,离子与波相互作用强,阻尼大。
\end{itemize}

波能
\begin{equation}
\pdv{\omega \varepsilon_\re}{\omega}
\approx 2 \ab(1 + \frac{1}{k^2 \lambda_{D,e}^2})
\end{equation}
长波情况下,由于场被屏蔽,动能远大于场能。

\section{朗道(Landau)阻尼}

\subsection{普莱姆(Plemelj)定理}

对含奇点的柯西型积分来说,当奇点趋于积分路径上时,有
\begin{equation}
\int_L \frac{f(z)}{z - z_0} \d z
= \PV\int_L \frac{f(z)}{z - z_0} \d z + \im\pi f(z_0)
\end{equation}

\subsection{朗道积分路径}

需要保证奇点在定义域内,

\subsection{介电函数中的积分}

\subsection{朗道阻尼的物理图像}

\subsubsection{电子等离子体波中的朗道阻尼}

\subsubsection{非线性朗道阻尼的物理图像}

速度与波相速度相近的粒子能够与波长时间地相互作用,发生共振。

\subsubsection{线性朗道阻尼的物理机制}

\subsection{朗道阻尼的推导}

\section{初值问题、弹道模及其物理意义}

\begin{equation}
\phi_k(\omega, \vector{k}) = \frac{\im}{k^2 \varepsilon(\omega, \vector{k})} \sum_\alpha q_\alpha \int \frac{f_{\alpha,k}(0)}{\omega - \vector{k} \cdot \vector{v}} \d^3 \vector{v}
\end{equation}
逆变换
\begin{equation}\begin{aligned}
\phi(t, \vector{k}) &=
\int_{-\infty+\im 0^+}^{+\infty+\im 0^+}
\phi_k(\omega, \vector{k}) \eu^{-\im\omega t} \d \omega \\
&= \frac{\im}{k^2} \sum_\alpha q_\alpha
\int f_{\alpha,k}(0) \int_{-\infty+\im 0^+}^{+\infty+\im 0^+}
    \frac{\eu^{-\im\omega t}}{
        \ab(\omega - \vector{k} \cdot \vector{v})
        \varepsilon(\omega, \vector{k})
    } \d \omega
\d^3 \vector{v} \\
&= \frac{\im}{k^2} \sum_\alpha q_\alpha
\int f_{\alpha,k}(0) \ab[
    \frac{
        \eu^{-\im \vector{k} \cdot \vector{v} t}
    }{\varepsilon(\vector{k} \cdot \vector{v}, \vector{k})}
    + \sum_j \frac{
        \eu^{-\im \omega_j t}
    }{
        \ab(\omega_j - \vector{k} \cdot \vector{v})
        \pdv{\varepsilon}!{\omega}|_{\omega_j}
    }
] \d^3 \vector{v}
\end{aligned}\end{equation}
其中,$\omega_j$ 为由色散方程 $\varepsilon(\omega_j, \vector{k}) = 0$ 得到的本征模。
$\omega = \vector{k} \cdot \vector{v}$ 为弹道模。

\section{电磁波模}

\subsection{速度空间各向同性下的电磁波模}

平衡态:$\vector{E}_0=\vector{0}$、$\vector{B}_0=\vector{0}$、$f_{\alpha,0}(\vector{r},\vector{v})=f_{\alpha,0}(v^2)$。

泊松方程
\begin{equation}\begin{gathered}
\dive{\vector{E}_1} = \frac{1}{\varepsilon_0}
\sum_\alpha q_\alpha \int f_{\alpha,1} \d^3\vector{v} \\
\vector{k} \cdot \vector{E}_k
= \frac{1}{\varepsilon_0}
\sum_\alpha q_\alpha \int f_{\alpha,k} \d^3\vector{v}
\end{gathered}\end{equation}

波动方程
\begin{equation}\begin{gathered}
\curl{\curl{\vector{E}_1}} =
- \frac{1}{c^2} \pdv[2]{\vector{E}_1}{t}
- \mu_0 \pdv*{\ab(
    \sum_\alpha q_\alpha \int \vector{v} f_{\alpha,1} \d^3\vector{v}
)}{t} \\
\ab(k^2 - \frac{\omega^2}{c^2}) \vector{E}_k
= \vector{E}_k\cdot\vector{k}\vector{k} + \im\mu_0\omega
    \sum_\alpha q_\alpha \int \vector{v} f_{\alpha,k} \d^3\vector{v}
\end{gathered}\end{equation}

弗拉索夫方程
\begin{equation}\begin{gathered}
\pdv{f_{\alpha,1}}{t}
+ \vector{v} \cdot \pdv{f_{\alpha,1}}{\vector{v}}
+ \frac{q_\alpha}{m_\alpha} \ab(\vector{E}_1 + \vector{v}\times\vector{B}_1)
\cdot \pdv{f_{\alpha,0}}{\vector{v}}
= 0 \\
f_{\alpha,k} = -\im \frac{q_\alpha}{m_\alpha}
\frac{\vector{E}_k}{\omega - \vector{k}\cdot\vector{v}}
\cdot \pdv{f_{\alpha,0}}{\vector{v}}
\end{gathered}\end{equation}
其中
\begin{equation}\begin{aligned}
\ab(\vector{v} \times \vector{B}_1) \cdot \pdv{f_{\alpha,0}}{\vector{v}}
&= \ab(\vector{v} \times \vector{B}_1)
\cdot \odv{f_{\alpha,0}}{v^2} \pdv{v^2}{\vector{v}} \\
&= \ab(\vector{v} \times \vector{B}_1)
\cdot 2 \odv{f_{\alpha,0}}{v^2} \vector{v} \\
&= 0
\end{aligned}\end{equation}
则
\begin{equation}\begin{aligned}
\im\mu_0\omega q_\alpha \int \vector{v} f_{\alpha,k} \d^3\vector{v}
&= \mu_0\omega \frac{q_\alpha^2}{m_\alpha}
\int \vector{v} \frac{\vector{E}_k}{\omega - \vector{k}\cdot\vector{v}}
\cdot \pdv{f_{\alpha,0}}{\vector{v}} \d^3\vector{v} \\
&= \omega \frac{\omega_{p,\alpha}^2}{c^2} \int \left[
\pdv{}{\vector{v}} \cdot \ab(
    \hat{f}_{\alpha,0} \vector{E}_k
    \frac{\vector{v}}{\omega - \vector{k}\cdot\vector{v}}
)
% \right. \\
% &\quad\left.
    - \hat{f}_{\alpha,0} \vector{E}_k \cdot
    \pdv*{\ab(\frac{\vector{v}}{\omega-\vector{k}\cdot\vector{v}})}{\vector{v}}
\right] \d^3\vector{v} \\
&= - \omega \frac{\omega_{p,\alpha}^2}{c^2} \left[
\vector{E}_k \int
\frac{\hat{f}_{\alpha,0}}{\omega - \vector{k}\cdot\vector{v}}
\d^3\vector{v}
+ \vector{k} \cdot \vector{E}_k \int
\frac{\vector{v} \hat{f}_{\alpha,0}}
{\ab(\omega - \vector{k}\cdot\vector{v})^2}
\d^3\vector{v}
\right]
\end{aligned}\end{equation}
代入波动方程得
\begin{equation}\begin{aligned}
\ab(\frac{\omega^2}{c^2} - k^2) \vector{E}_k
+ \vector{E}_k\cdot\vector{k}\vector{k}
= \omega \sum_\alpha \frac{\omega_{p,\alpha}^2}{c^2}
\left[
\vector{E}_k \int
\frac{\hat{f}_{\alpha,0}}{\omega - \vector{k}\cdot\vector{v}}
\d^3\vector{v}
+ \vector{k} \cdot \vector{E}_k \int
\frac{\vector{v} \hat{f}_{\alpha,0}}
{\ab(\omega - \vector{k}\cdot\vector{v})^2}
\d^3\vector{v}
\right]
\end{aligned}\end{equation}
考虑 $\perp\vector{k}$ 方向的波动方程:
\begin{equation}\begin{aligned}
\ab(\frac{\omega^2}{c^2} - k^2) \vector{k}\times\vector{E}_k
= \omega \sum_\alpha \frac{\omega_{p,\alpha}^2}{c^2}
\left[
\vector{k}\times\vector{E}_k \int
\frac{\hat{f}_{\alpha,0}}{\omega - \vector{k}\cdot\vector{v}}
\d^3\vector{v}
+ \vector{k} \cdot \vector{E}_k \int
\frac{\vector{k}\times\vector{v} \hat{f}_{\alpha,0}}
{\ab(\omega - \vector{k}\cdot\vector{v})^2}
\d^3\vector{v}
\right]
\end{aligned}\end{equation}
其中
\begin{equation}\begin{aligned}
\int
\frac{\vector{k}\times\vector{v} \hat{f}_{\alpha,0}}
{\ab(\omega - \vector{k}\cdot\vector{v})^2}
\d^3\vector{v}
&= \int
\frac{\d v_\parallel}{\ab(\omega/k - v_\parallel)^2}
\int \vector{v}_\perp \hat{f}_{\alpha,0}\ab(v_\perp^2 + v_\parallel^2)
\d^2\vector{v}_\perp \\
&= 0
\end{aligned}\end{equation}
则 $\perp\vector{k}$ 方向的波动方程可化简为
\begin{equation}
\ab[
    1 - \frac{k^2 c^2}{\omega^2}
    - \sum_\alpha \frac{\omega_{p,\alpha}^2}{\omega}
    \int
    \frac{\hat{f}_{\alpha,0}}{\omega - \vector{k}\cdot\vector{v}}
    \d^3\vector{v}
]
\vector{k}\times\vector{E}_k = 0
\end{equation}
考虑到 $\omega/k \sim c \gg v_\parallel$,则色散关系可近似为
\begin{equation}
\omega^2 = k^2 c^2 + \sum_\alpha \omega_{p,\alpha}^2
\end{equation}

\subsection{速度空间各向异性下的电磁波模 —— Weibel 不稳定性}

平衡分布 $f_{\alpha,0} = f_{\alpha,0}(v_\parallel^2, v_\perp^2)$。
则 $\ab(\vector{v}\times\vector{B}_1)\cdot\pdv{f_{\alpha,0}}{\vector{v}}\neq0$。

磁场扰动
\begin{equation}
\vector{B}_k = \frac{\vector{k} \times \vector{E}_k}{\omega}
\end{equation}
则弗拉索夫方程变为
\begin{equation}\begin{gathered}
\pdv{f_{\alpha,1}}{t}
+ \vector{v} \cdot \pdv{f_{\alpha,1}}{\vector{v}}
+ \frac{q_\alpha}{m_\alpha} \ab(\vector{E}_1 + \vector{v}\times\vector{B}_1)
\cdot \pdv{f_{\alpha,0}}{\vector{v}}
= 0 \\
\begin{aligned}
f_{\alpha,k} &= -\im \frac{q_\alpha}{m_\alpha}
\frac{1}{\omega - \vector{k}\cdot\vector{v}} \ab[
\vector{E}_k + \vector{v} \times
\ab(\frac{\vector{k}}{\omega} \times \vector{E}_k)]
\cdot \pdv{f_{\alpha,0}}{\vector{v}} \\
&= -\im \frac{q_\alpha}{\omega m_\alpha} \ab(
    \vector{E}_k + \frac{\vector{E}_k \cdot \vector{v}}
    {\omega - \vector{k}\cdot\vector{v}} \vector{k}
) \cdot \pdv{f_{\alpha,0}}{\vector{v}}
\end{aligned}
\end{gathered}\end{equation}
则
\begin{equation}\begin{gathered}\begin{aligned}
\im\mu_0\omega q_\alpha \int \vector{v} f_{\alpha,k} \d^3\vector{v}
&= \mu_0 \frac{q_\alpha^2}{m_\alpha}
\int \vector{v} \ab(
    \vector{E}_k + \frac{\vector{E}_k \cdot \vector{v}}
    {\omega - \vector{k}\cdot\vector{v}} \vector{k}
)
\cdot \pdv{f_{\alpha,0}}{\vector{v}} \d^3\vector{v} \\
&= - \frac{\omega_{p,\alpha}^2}{c^2} \left[
\int \hat{f}_{\alpha,0} \vector{E}_k \cdot \pdv{\vector{v}}{\vector{v}} \d^3\vector{v}
+ \int \hat{f}_{\alpha,0} \vector{k} \cdot
    \pdv*{\ab(\frac{\vector{E}_k \cdot \vector{v}\vector{v}}
    {\omega-\vector{k}\cdot\vector{v}})}{\vector{v}}
\d^3\vector{v}
\right]\end{aligned} \\
= - \frac{\omega_{p,\alpha}^2}{c^2} \left[ \vector{E}_k
+ \vector{k} \cdot \vector{E}_k \int
    \frac{\vector{v} \hat{f}_{\alpha,0} \d^3\vector{v}}
    {\omega-\vector{k}\cdot\vector{v}}
+ \vector{k} \vector{E}_k \cdot \int
    \frac{\vector{v} \hat{f}_{\alpha,0} \d^3\vector{v}}
    {\omega-\vector{k}\cdot\vector{v}}
+ k^2 \vector{E}_k \cdot \int
    \frac{\vector{v}\vector{v} \hat{f}_{\alpha,0} \d^3\vector{v}}
    {\ab(\omega-\vector{k}\cdot\vector{v})^2}
\right]
\end{gathered}\end{equation}
代入波动方程得
\begin{equation}\begin{gathered}
\ab(\frac{\omega^2}{c^2} - k^2) \vector{E}_k
+ \vector{E}_k \cdot \vector{k}\vector{k}
= \\
\sum_\alpha \frac{\omega_{p,\alpha}^2}{c^2}
\left[ \vector{E}_k
+ \vector{k} \cdot \vector{E}_k \int
    \frac{\vector{v} \hat{f}_{\alpha,0} \d^3\vector{v}}
    {\omega-\vector{k}\cdot\vector{v}}
+ \vector{k} \vector{E}_k \cdot \int
    \frac{\vector{v} \hat{f}_{\alpha,0} \d^3\vector{v}}
    {\omega-\vector{k}\cdot\vector{v}}
+ k^2 \vector{E}_k \cdot \int
    \frac{\vector{v}\vector{v} \hat{f}_{\alpha,0} \d^3\vector{v}}
    {\ab(\omega-\vector{k}\cdot\vector{v})^2}
\right]
\end{gathered}\end{equation}
考虑 $\perp\vector{k}$ 方向的波动方程:
\begin{equation}\begin{gathered}
\ab(\frac{\omega^2}{c^2} - k^2) \vector{k}\times\vector{E}_k
= \\
\sum_\alpha \frac{\omega_{p,\alpha}^2}{c^2}
\left[ \vector{k}\times\vector{E}_k
+ \vector{k} \cdot \vector{E}_k \int
    \frac{\vector{k}\times\vector{v} \hat{f}_{\alpha,0}}
    {\omega-\vector{k}\cdot\vector{v}} \d^3\vector{v}
+ k^2 \vector{k}\times \ab(\vector{E}_k \cdot \int
    \frac{\vector{v}\vector{v} \hat{f}_{\alpha,0} \d^3\vector{v}}
    {\ab(\omega-\vector{k}\cdot\vector{v})^2})
\right] \\
\ab(\frac{\omega^2}{c^2} - k^2
- \sum_\alpha \frac{\omega_{p,\alpha}^2}{c^2})
\vector{k} \times \vector{E}_k
= \sum_\alpha \frac{\omega_{p,\alpha}^2}{c^2}k^2 \int
    \frac{\vector{E}_k \cdot \vector{v} \vector{k} \times \vector{v}}
    {\ab(\omega-\vector{k}\cdot\vector{v})^2}
\hat{f}_{\alpha,0} \d^3\vector{v}
\end{gathered}\end{equation}
得
\begin{equation}
\omega^2 - k^2 c^2 - \sum_\alpha \omega_{p,\alpha}^2 \ab(
    1 + \frac{k^2}{\omega^2} \ab<v_{\alpha\perp}^2>
)
\end{equation}
令
\begin{subequations}\begin{align}
\omega_p^2 &= \sum_\alpha \omega_{p,\alpha}^2 \\
\ab<v_\perp^2> &= \frac{1}{\omega_p^2}
    \sum_\alpha \omega_{p,\alpha}^2 \ab<v_{\alpha\perp}^2>
\end{align}\end{subequations}
得
\begin{equation}
\omega^2 = \frac{k^2 c^2 + \omega_p^2}{2} \ab[
    1 \pm \sqrt{1 + 
        \frac{4 k^2 \omega_p^2 \ab<v_\perp^2>}
        {\ab(k^2 c^2 + \omega_p^2)^2}
}]
\end{equation}
有一对实根与一对纯虚根。
当 $k \to 0$ 时,纯虚根 $\Im(\omega) \to 0$。
当 $k \gg \omega_p/c$ 时,纯虚根 $\Im(\omega) \approx \omega_p \sqrt{\ab<v_\perp^2>}/c$。
