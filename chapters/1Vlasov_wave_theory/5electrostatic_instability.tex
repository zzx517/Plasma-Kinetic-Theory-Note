
\chapter{静电不稳定性}

不稳定性条件:
\begin{enumerate}
    \item 多余的自由能,体系处于非热平衡态;
    \item 正反馈机制。
\end{enumerate}

自由能来源:
\begin{itemize}
    \item 外部输入;
    \item 内部偏离热平衡态。
\end{itemize}

不稳定性机制
\begin{itemize}
    \item 波-波相互作用 —— 内部波模耦合(反应型)、外部波(参量过程);
    \item 波-粒子相互作用 —— 逆朗道阻尼效应。
\end{itemize}

共振条件:
\begin{enumerate}
    \item $\omega = \vec{k}\cdot\vec{v}$
    \item $\omega - k_\parallel v_\parallel = n \omega_{c,\alpha}$
    \item $\omega_1 \approx \omega_2$
    \item $\omega_1 \pm \omega_2 \approx \omega_3$
\end{enumerate}

\section{损失锥型速度分布引起的静电不稳定性}

约束条件:
\begin{equation}
\frac{v_\perp^2}{v^2} \geq \frac{B_{\min}}{B_{\max}}
\end{equation}

\section{各向异性分布引起的不稳定性}

\subsection{电子静电波不稳定性}

\subsection{离子声波不稳定性}

\section{定向漂移引起的不稳定性}

\subsection{束-等离子体不稳定性}

\subsection{离子漂移不稳定性}

\subsection{电流驱动的静电离子回旋波不稳定性}

\subsection{非均匀等离子体中低频漂移波不稳定性}
