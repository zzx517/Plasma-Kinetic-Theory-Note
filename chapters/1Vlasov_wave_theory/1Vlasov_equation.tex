
\chapter{波动的物理特点和数学描述}

\section{弗拉索夫方程组}

分布函数 $f_\alpha(t, \vector{r}, \vector{v})$ 代表了在 $t$ 时刻、位置在体积元 $\d^3\vector{r}$ 范围内、速度在体积元 $\d^3\vector{v}$ 范围内的 $\alpha$ 粒子数目。也可以使用它的归一化形式:
\begin{subequations}\begin{align}
\hat{f}_\alpha(t, \vector{r}, \vector{v}) &
= \frac{f_\alpha(t, \vector{r}, \vector{v})}
{n_\alpha(t, \vector{r})} \\
n_\alpha(t, \vector{r}) &= \int f_\alpha(t, \vector{r}, \vector{v}) \d^3\vector{v}
\end{align}\end{subequations}

分布函数的时间演化,是由动理学方程和麦克斯韦方程组联合描述:
\begin{subequations}\begin{gather}
\pdv{f_\alpha}{t}
+ \vector{v} \cdot \pdv{f_\alpha}{\vector{r}}
+ \frac{q_\alpha}{m_\alpha} \ab(
    \vector{E} + \vector{v} \times \vector{B}
) \cdot \pdv{f_\alpha}{\vector{v}}
= \ab(\pdv{f_\alpha}{t})_c \\
\dive{\vector{E}} = \frac{1}{\varepsilon_0}
\sum_\alpha q_\alpha \int f_\alpha \, \dif^3 \vector{v}
+ \frac{\rho_{\text{ext}}}{\varepsilon_0} \\
\curl{\vector{E}} = - \pdv{\vector{B}}{t} \\
\dive{\vector{B}} = 0 \\
\curl{\vector{B}} = \varepsilon_0 \mu_0 \pdv{\vector{E}}{t}
+ \mu_0 \sum_\alpha q_\alpha \int \vector{v} f_\alpha \, \dif^3 \vector{v}
+ \mu_0 \vector{J}_{\text{ext}}
\end{gather}\end{subequations}
其中,$\rho_{\text{ext}}$ 与 $\vector{J}_{\text{ext}}$ 是外加的电荷密度与电流密度。$\ab(\pdv{f_\alpha}/{t})_c$ 是碰撞引起的分布函数变化速率,在无碰撞的情况下,此项可以忽略,此时,上述方程组就称为 \textbf{弗拉索夫方程组}。

\subsubsection{线性问题及其分类}

与之前类似,我们可以将物理量根据时空的变化速率的快慢分为零阶量和一阶量:
\begin{subequations}\begin{align}
f_\alpha(t, \vector{r}, \vector{v}) &= f_{\alpha,0}(t, \vector{r}, \vector{v}) + f_{\alpha,1}(t, \vector{r}, \vector{v}) \\
\vector{E}(t, \vector{r}) &= \vector{E}_0(t, \vector{r}) + \vector{E}_1(t, \vector{r}) \\
\vector{B}(t, \vector{r}) &= \vector{B}_0(t, \vector{r}) + \vector{B}_1(t, \vector{r})
\end{align}\end{subequations}
且一阶量远小于零阶量
\begin{equation}
f_{\alpha,1} \ll f_{\alpha,0}, \quad
\vector{E}_1 \ll \vector{E}_0, \quad
\vector{B}_1 \ll \vector{B}_0
\end{equation}

这样,我们可以将总演化方程组分开写成零阶方程
\begin{subequations}\begin{gather}
\pdv{f_{\alpha,0}}{t}
+ \vector{v} \cdot \pdv{f_{\alpha,0}}{\vector{r}}
+ \frac{q_\alpha}{m_\alpha} \big(
    \vector{E}_0 + \vector{v} \times \vector{B}_0
\big) \cdot \pdv{f_{\alpha,0}}{\vector{v}}
= \left( \pdv{f_{\alpha,0}}{t} \right)_c \\
\dive{\vector{E}_0} = \frac{1}{\varepsilon_0}
\sum_\alpha q_\alpha \int f_{\alpha,0} \, \dif^3 \vector{v}
+ \frac{\rho_{\text{ext}}}{\varepsilon_0} \\
\curl{\vector{E}_0} = - \pdv{\vector{B}_0}{t} \\
\dive{\vector{B}_0} = 0 \\
\curl{\vector{B}_0} = \varepsilon_0 \mu_0 \pdv{\vector{E}_0}{t}
+ \mu_0 \sum_\alpha q_\alpha \int \vector{v} f_{\alpha,0} \, \dif^3 \vector{v}
+ \mu_0 \vector{J}_{\text{ext}}
\end{gather}\end{subequations}
和一阶方程
\begin{subequations}\begin{gather}
\pdv{f_{\alpha,1}}{t}
+ \vector{v} \cdot \pdv{f_{\alpha,1}}{\vector{r}}
+ \frac{q_\alpha}{m_\alpha} \big(
    \vector{E}_0 + \vector{v} \times \vector{B}_0
\big) \cdot \pdv{f_{\alpha,1}}{\vector{v}}
+ \frac{q_\alpha}{m_\alpha} \big(
    \vector{E}_1 + \vector{v} \times \vector{B}_1
\big) \cdot \pdv{f_{\alpha,0}}{\vector{v}}
= \left( \pdv{f_{\alpha,1}}{t} \right)_c \\
\dive{\vector{E}_1} = \frac{1}{\varepsilon_0}
\sum_\alpha q_\alpha \int f_{\alpha,1} \, \dif^3 \vector{v} \\
\curl{\vector{E}_1} = - \pdv{\vector{B}_1}{t} \\
\dive{\vector{B}_1} = 0 \\
\curl{\vector{B}_1} = \varepsilon_0 \mu_0 \pdv{\vector{E}_1}{t}
+ \mu_0 \sum_\alpha q_\alpha \int \vector{v} f_{\alpha,1} \, \dif^3 \vector{v}
\end{gather}\end{subequations}

在比较简单的情况下,平衡分布 $f_{\alpha,0}$ 往往取热力学平衡分布——麦克斯韦分布:
\begin{equation} \label{eq:Maxwell_distribution} \begin{aligned}
\hat{f}_{\text{M}, \alpha}
&= \left(\frac{m_\alpha}{2 \pi T_\alpha}\right)^{3/2}
\exp{\left[- \frac{m_\alpha}{2 T_\alpha} (\vector{v} - \vector{u}_\alpha)^2\right]} \\
&= \frac{1}{\sqrt{\pi^3} v_{t, \alpha}^3}
\exp{\left[- \frac{(\vector{v} - \vector{u}_\alpha)^2}{v_{t, \alpha}^2}\right]}
\end{aligned}\end{equation}
其中,$\vector{u}_\alpha$ 为粒子宏观流体速度,
\begin{equation}
v_{t, \alpha} = \sqrt{\frac{2 T_\alpha}{m_\alpha}}
\end{equation}
为粒子热速度。

线性波动问题分类

波动问题求解方法

\subsection{常用数学处理方法}

\subsubsection{傅里叶(FourierT)变换}

傅里叶变换及其逆变换
\begin{subequations}\begin{align}
\FourierT[f](\omega) &= \int_{-\infty}^{+\infty} f(t) \eu^{- \im \omega t} \, \dif t \\
\FourierTinv[f](t) &= \int_{-\infty}^{+\infty} f(\omega) \eu^{\im \omega t} \, \dif \omega
\end{align}\end{subequations}

\begin{subequations}\begin{align}
\FourierT[\FourierT[f]](t) &= f(-t) \\
\FourierTinv[\FourierTinv[f]](t) &= f(-t) \\
\FourierT[f(t - t_0)] &= \eu^{-\im \omega t_0} \FourierT[f](\omega) \\
\FourierTinv[f(\omega - \omega_0)] &= \eu^{\im \omega_0 t} \FourierTinv[f](t) \\
\FourierT[f(t / \lambda)] &= \lambda \FourierT[f](\lambda \omega) \\
\FourierTinv[f(\omega / \lambda)] &= \lambda \FourierTinv[f](\lambda t) \\
\FourierT\left[\DerivN{f}{t}{m}\right] &= \left(\im\omega\right)^m \FourierT[f](\omega) \\
\FourierTinv\left[\DerivN{f}{\omega}{m}\right] &= \left(-\im t\right)^m \FourierTinv[f](t) \\
\FourierT[f * g] &= \FourierT[f] \FourierT[g] \\
\FourierTinv[f * g] &= \FourierTinv[f] \FourierTinv[g]
\end{align}\end{subequations}

\subsubsection{拉普拉斯(Laplace)变换}

\begin{equation}\begin{aligned}
f(t) &\LaplaceTs \int_0^{+\infty} f(t) \eu^{- s t} \d t \\
- \frac{\im}{2\pi}\int_{c-\im\infty}^{c+\im\infty} F(s) \eu^{- s t} \d s &\LaplaceTs F(s)\\
\end{aligned}\end{equation}

\subsection{波动的表达方式}

平面单色波:
\begin{equation}
f(t,\vector{r}) = f_A \eu^{\im\ab(\vector{k}\cdot\vector{r} - \omega t)}
\end{equation}
定义波的相速度为
\begin{equation}
v_p = \frac{\omega}{k}
\end{equation}
波的群速度为
\begin{equation}
v_g = \odv{\omega}{k}
\end{equation}

取频率 $\omega$ 为复数,则可以表示波的不稳定/阻尼:
\begin{equation}
f(t,\vector{r}) = f_A \eu^{\Im(\omega) t} \eu^{\im\ab[\vector{k}\cdot\vector{r} - \Re(\omega)] t)}
\end{equation}
当 $\Im(\omega)>0$ 时,波会增长;当 $\Im(\omega)<0$ 时,波会阻尼。

\section{扰动波场的能量传播方程}

\subsection{坡印廷(Poynting)定理}

电磁场的能量传播由坡印廷(Poynting)定理描述:
\begin{subequations}\begin{gather}
\pdv{w}{t} + \dive{\vector{S}} + \vector{E} \cdot \vector{J} = 0 \\
w = \frac{\varepsilon_0}{2} \vector{E}^2 + \frac{1}{2\mu_0} \vector{B}^2 \\
\vector{S} = \frac{1}{\mu_0} \vector{E} \times \vector{B}
\end{gather}\end{subequations}
其中,$w$ 为电磁场的能量密度,$\vector{S}$ 为电磁场的能流密度,也被称作坡印廷矢量。

在静态或准静态过程中,欧姆定律为
\begin{equation}
\vector{J} = \sigma \vector{E}
\end{equation}
其中,$\sigma$ 为电导率。

而一般情况下,有
\begin{equation}\begin{aligned}
\vector{J}(t, \vector{r}) &= \vector{\sigma}(t, \vector{r}) \cdot * \vector{E}(t, \vector{r}) \\
\vector{J}(\omega, \vector{k}) &= \vector{\sigma}(\omega, \vector{k}) \cdot \vector{E}(\omega, \vector{k})
\end{aligned}\end{equation}

\subsection{波的能量传播方程}

对于准平面单色波,扰动电场可写成
\begin{equation}
\vector{E}(t) = \frac{1}{2} \ab[
    \vector{E}_0(t) \eu^{-\im\omega_0 t)}
  + \vector{E}_0^*(t) \eu^{\im\omega_0 t)}
]
\end{equation}
其中,$\omega_0$ 为实数,$\vector{E}^*$ 代表 $\vector{E}$ 的复共轭。$\vector{E}_0$ 代表扰动电场的慢变部分($\pdv{\vector{E}_0}!{t}\ll\omega_0\vector{E}_0$)。

扰动电场的傅里叶变换为
\begin{equation}\begin{aligned}
\vector{E}(\omega) &= \int \vector{E}(t) \eu^{\im \omega t} \d t \\
&= \int \frac{1}{2} \ab[
    \vector{E}_0(t) \eu^{-\im\omega_0 t)}
  + \vector{E}_0^*(t) \eu^{\im\omega_0 t)}
] \eu^{\im \omega t} \d t \\
&= \frac{1}{2} \ab[
    \vector{E}_0(\omega - \omega_0)
  + \vector{E}_0^*(\omega + \omega_0)
]
\end{aligned}\end{equation}

则频域里的扰动电流为
\begin{equation}\begin{aligned}
\vector{J}(\omega) &= \vector{\sigma}(\omega) \cdot \vector{E}(\omega) \\
&= \frac{1}{2} \vector{\sigma}(\omega) \cdot \ab[
    \vector{E}_0(\omega - \omega_0)
  + \vector{E}_0^*(\omega + \omega_0)
]
\end{aligned}\end{equation}

逆傅里叶变换后,得
\begin{equation}\begin{aligned}
\vector{J}(t) &= \frac{1}{2\pi} \int \vector{J}(\omega) \eu^{-\im\omega t} \d\omega \\
&= \frac{1}{2} \ab[
    \ab(
        \eval{\vector{\sigma}}_{\omega_0}
        + \eval{\pdv{\vector{\sigma}}{\omega}}_{\omega_0} \im \pdv{}{t}
    ) \cdot \vector{E}_0(t) \eu^{-\im\omega_0 t)}
  + \ab(
        \eval{\vector{\sigma}}_{-\omega_0}
        + \eval{\pdv{\vector{\sigma}}{\omega}}_{-\omega_0} \im \pdv{}{t}
    ) \cdot \vector{E}_0^*(t) \eu^{\im\omega_0 t)}
] \\
&= \frac{1}{2} \ab[
    \eval{\vector{\sigma}}_{\omega_0 + \im \pdv{}!{t}}
    \cdot \vector{E}_0(t) \eu^{-\im\omega_0 t)}
  + \eval{\vector{\sigma}}_{-\omega_0 + \im \pdv{}!{t}}
    \cdot \vector{E}_0^*(t) \eu^{\im\omega_0 t)}
]
\end{aligned}\end{equation}
考虑 $\vector{J}(t)$ 为实数,则
\begin{subequations}\begin{align}
\eval{\vector{\sigma}^*}_{\omega_0 + \im \pdv{}!{t}} &
= \eval{\vector{\sigma}}_{-\omega_0 + \im \pdv{}!{t}} \\
\eval{\vector{\sigma}^*}_{-\omega_0 + \im \pdv{}!{t}} &
= \eval{\vector{\sigma}}_{\omega_0 + \im \pdv{}!{t}}
\end{align}\end{subequations}

在快变周期 $2\pi/\omega_0$ 内,对 $\vector{E}\cdot\vector{J}$ 作平均:
\begin{equation}\begin{aligned}
\ab<\vector{E}\cdot\vector{J}> &= \frac{1}{4} \ab[
    \vector{E}_0(t) \cdot \vector{J}_0^*(t)
    + \vector{E}_0^*(t) \cdot \vector{J}_0(t)
] \\
&= \frac{1}{4} \vector{E}_0^*(t) \cdot \peval{
    \vector{\sigma} + \vector{\sigma}^\HT
}_{\omega_0} \cdot \vector{E}_0
+ \frac{\im}{8} \pdv*{\ab[\vector{E}_0^*(t) \cdot \eval{
    \pdv{\ab(\vector{\sigma} - \vector{\sigma}^\HT)}{\omega}
}_{\omega_0} \cdot \vector{E}_0]}{t} \\
&= \frac{1}{2} \vector{E}_0^*(t) \cdot \eval{
    \vector{\sigma}_\re
}_{\omega_0} \cdot \vector{E}_0
- \frac{1}{4} \pdv*{\ab[\vector{E}_0^*(t) \cdot \eval{
    \pdv{\vector{\sigma}_\im}{\omega}
}_{\omega_0} \cdot \vector{E}_0]}{t}
\end{aligned}\end{equation}
其中,$\HT$ 代表厄米共轭(转置 $\T$ + 复共轭),
$\vector{\sigma}_\re=\ab(\vector{\sigma}+\vector{\sigma}^\HT)/2$代表厄米部分,
$\vector{\sigma}_\im=\ab(\vector{\sigma}-\vector{\sigma}^\HT)/2\im$代表反厄米部分。

定义相对介电函数张量:
\begin{subequations}\begin{align}
\vector{\varepsilon} &= \vector{I} + \frac{\im}{2\varepsilon_0 \omega} \vector{\sigma} \\
\vector{\varepsilon}_\re &= \vector{I} - \frac{\vector{\sigma}_\im}{2\varepsilon_0 \omega} \\
\vector{\varepsilon}_\im &= \frac{\vector{\sigma}_\re}{2\varepsilon_0 \omega} \\
\end{align}\end{subequations}

则电磁波的能量传播方程可写作
\begin{subequations}\begin{gather}
\pdv{w_\re} + \dive{\vector{S}} + w_\im = 0 \\
w_\re = \frac{1}{2\mu_0} \vector{B}_0^2
    + \frac{\varepsilon_0}{2} \vector{E}_0^*
        \cdot \eval{\pdv{\omega \vector{\varepsilon}_\re}{\omega}}_{\omega_0}
        \cdot \vector{E}_0 \\
w_\im = \varepsilon_0 \omega \vector{E}_0^*
        \cdot \eval{\vector{\varepsilon}_\im}_{\omega_0}
        \cdot \vector{E}_0 \\
\vector{S} = \frac{1}{\mu_0} \ab(
    \vector{E}_0 \times \vector{B}_0^*
    + \vector{E}_0^* \times \vector{B}_0
)
\end{gather}\end{subequations}

若仅考虑静电波,则有
\begin{subequations}\begin{gather}
\pdv{w^\text{ES}_\re} + w^\text{ES}_\im = 0 \\
w^\text{ES}_\re = \frac{\varepsilon_0}{2} \vector{E}_0^*
        \cdot \eval{\pdv{\omega \vector{\varepsilon}_\re}{\omega}}_{\omega_0}
        \cdot \vector{E}_0
= \frac{\varepsilon_0}{2} E_0^2 \eval{\pdv{\omega \varepsilon_\re}{\omega}}_{\omega_0 }\\
w^\text{ES}_\im = \varepsilon_0 \omega \vector{E}_0^*
        \cdot \eval{\vector{\varepsilon}_\im}_{\omega_0}
        \cdot \vector{E}_0
= \varepsilon_0 E_0^2 \omega_0 \eval{\varepsilon_\im}_{\omega_0}
\end{gather}\end{subequations}
可以解得
\begin{equation}
E_0^2 \propto \eu^{2 \gamma t}, \quad
\gamma = - \eval{\frac{\omega \varepsilon_\im}{
    \pdv{\ab(\omega \varepsilon_\re)}!{\omega}
}}_{\omega_0}
\end{equation}

\begin{description}
    \item[正/负能波] $\pdv{\ab(\omega \varepsilon_\re)}!{\omega}$ 的符号;
    \item[正/负耗散] $\varepsilon_\im$ 的符号。
\end{description}

\subsection{同时考虑时间和空间平均的能量方程}
