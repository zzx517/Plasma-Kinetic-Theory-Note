
\chapter{均匀外磁场、空间均匀的线性波动问题}

平衡量:
$\vec{E}_0 = \vec{0}$、
$\vec{B}_0 = \text{const}$、
$f_{\alpha,0} = f_{\alpha,0}(\vec{v})$。

线性化弗拉索夫方程:
\begin{equation}
\ab[
    \pdv{}{t}
    + \vec{v} \cdot \pdv{}{\vec{r}}
    + \frac{q_\alpha}{m_\alpha} \ab(\vec{v} \times \vec{B}_0)
] f_{\alpha,1}
+ \frac{q_\alpha}{m_\alpha}
\ab(\vec{E}_1 + \vec{v} \times \vec{B}_1)
\cdot \pdv{f_{\alpha, 0}}{\vec{v}}
= 0
\end{equation}

\section{沿未扰动粒子轨道积分法}

\subsection{粒子的未扰动轨道及弗拉索夫方程}

\subsubsection{相空间的欧拉坐标与拉格朗日坐标}

欧拉坐标:
\begin{equation}\begin{cases}
t & \Rightarrow \text{时间} \\
\vec{r} & \Rightarrow \text{位置空间} \\
\vec{v} & \Rightarrow \text{速度空间}
\end{cases}\end{equation}

拉格朗日坐标:
\begin{equation}\begin{cases}
t & \Rightarrow \text{时间} \\
\vec{r}_0 & \Rightarrow \text{粒子的初始位置} \\
\vec{v}_0 & \Rightarrow \text{粒子的初始速度}
\end{cases}\end{equation}

存在相互转换
\begin{equation}\begin{aligned}
\vec{r} &= \vec{r}(t, \vec{r}_0, \vec{v}_0) \\
\vec{v} &= \mdv{\vec{r}}{t}(t, \vec{r}_0, \vec{v}_0)
\end{aligned}\end{equation}
其中,$\mdv{}{t}$ 为随流导数。

\subsubsection{粒子的未扰动轨道}

\begin{equation}
\mdv[2]{\vec{r}}{t} = \frac{1}{m_\alpha} \vec{F}\ab(t, \vec{r}, \mdv{\vec{r}}{t})
= \frac{q_\alpha}{m_\alpha} \mdv{\vec{r}}{t} \times \vec{B}_0
\end{equation}

\subsubsection{}

\begin{equation}
\mdv{f_\alpha}{t} = \pdv{f_\alpha}{t}
+ \vec{v} \cdot \pdv{f_\alpha}{\vec{r}}
+ \frac{q_\alpha}{m_\alpha} \ab(\vec{E} + \vec{v} \textbf{}) \cdot \pdv{f_\alpha}{\vec{r}}
= 0
\end{equation}
线性化
\begin{equation}
\mdv{f_\alpha}{t}
+ \frac{q_\alpha}{m_\alpha}
\ab(\vec{E}_1 + \vec{v} \times \vec{B}_1)
\cdot \pdv{f_{\alpha, 0}}{\vec{v}}
= 0
\end{equation}

\subsection{弗拉索夫方程的积分}

将拉格朗日坐标下的线性化弗拉索夫方程对时间 $t$ 积分:
\begin{equation}
f_{\alpha,1}(t, \vec{r}_0, \vec{v}_0)
= f_{\alpha,1}(t=0, \vec{r}_0, \vec{v}_0)
- \frac{q_\alpha}{m_\alpha} \int_0^t
    \ab(\vec{E}_1 + \vec{v} \times \vec{B}_1)
    \cdot \pdv{f_{\alpha, 0}}{\vec{v}}
\d\tau
\end{equation}
再将拉格朗日坐标 $(t, \vec{r}_0, \vec{v}_0)$ 变换到欧拉坐标 $(t, \vec{r}, \vec{v})$,就解出了线性化弗拉索夫方程。对于本征值问题,初值 $f_{\alpha,1}(t=0, \vec{r}_0, \vec{v}_0) = 0$。

\subsection{电导张量与电磁张量}

波动方程
\begin{equation}
\laplacian{\vec{E}_1} - \grad{\dive{\vec{E}_1}}
= - \frac{1}{c^2} \pdv[2]{\vec{E}_1}{t} - \mu_0 \pdv{\vec{J}_1}{t}
\end{equation}
变换
\begin{equation}
k^2 \vec{E}_k - \vec{k}\vec{k} \cdot \vec{E}_k
= \frac{\omega^2}{c^2} \vec{E}_k + \im \mu_0 \omega \vec{J}_k
\end{equation}

电导张量
\begin{equation}
\vec{J}_k = \vec{\sigma}(\omega, \vec{k}) \cdot \vec{E}_k
\end{equation}

介电张量
\begin{equation}
\vec{\varepsilon}(\omega, \vec{k})
= \varepsilon_0 \vec{I}
+ \frac{\im}{\omega} \vec{\sigma}(\omega, \vec{k})
\end{equation}

电磁张量
\begin{equation}
\vec{D}(\omega, \vec{k})
= \vec{k}\vec{k} - k^2 \vec{I}
+ \omega^2 \mu_0 \varepsilon(\omega, \vec{k})
\end{equation}

色散方程
\begin{equation}
\det\ab(\vec{D}) = 0
\end{equation}

扰动电流
\begin{equation}
\vec{J}_{\alpha,k} = q_\alpha n_{\alpha,0}
\int \vec{v} f_{\alpha,k} \d^3\vec{v}
\end{equation}
得
\begin{equation}
\vec{\sigma}_\alpha = - \varepsilon_0 \frac{\omega_{p,\alpha}^2}{\omega}
\int \d^3\vec{v} \int_0^{t} \d\tau
\vec{v} \pdv{\hat{f}_{\alpha,0}}{\dot{\vec{\xi}}} \cdot \ab[
    \ab(\omega - \vec{k}\cdot\dot{\vec{\xi}}) \vec{I}
    + \vec{k} \dot{\vec{\xi}}
] \eu^{\im \ab(\vec{k} \cdot \vec{\xi} - \omega \tau)}
\end{equation}

极化率张量
\begin{equation}
\vec{\chi}_\alpha = \frac{\im}{\omega \varepsilon_0} \vec{\sigma}
\end{equation}

\subsubsection{电磁张量表达式}

\subsubsection{讨论}

波-粒子相互作用

有限拉莫(Larmor)轨道效应:在色散方程中出现了贝塞尔函数的无穷求和,自变量为 $k_\perp r_c$,它们会使波的色散关系产生重大改变。


\subsection{均匀、磁化等离子体色散方程的各种表达式}

\section{沿未扰动导心轨道积分法}

粒子的未扰动轨道
\begin{equation}
\vec{r} = \vec{r}_g + \vec{r}_c
\end{equation}

拉莫回旋运动速度
\begin{equation}
\vec{v}_c = - \omega_c \hat{\vec{b}} \times \vec{r}_c
\end{equation}
则
\begin{equation}
\vec{r}_c = - \frac{\vec{v}_c \times \hat{\vec{b}}}{\omega_c}
\end{equation}
则
\begin{equation}
\vec{r}_g = \vec{r} + \frac{\vec{v}_c \times \hat{\vec{b}}}{\omega_c}
\end{equation}


\subsubsection{准静电与准电磁模近似及其条件}

波矢 $\vec{k} = k \hat{\vec{k}}$。
扰动电场 $\vec{E}_1 = E_l \vec{e}_l + E_t \vec{e}_k$。

电磁张量
\begin{equation}
\vec{D} = \frac{k^2 c^2}{\omega^2} \ab(\hat{\vec{k}}\hat{\vec{k}}
- \vec{I}) + \vec{\varepsilon}
= \begin{pmatrix}
    \varepsilon_{ll} & \varepsilon_{lt} \\
    \varepsilon_{tl} & \varepsilon_{tt} - k^2 c^2 / \omega^2
\end{pmatrix}
\end{equation}

纯静电条件 $\varepsilon_{lt} = 0$,色散关系 $\varepsilon_{ll}(\omega) = 0$。

准静电条件 $E_l \gg E_t, \varepsilon_{ll} \gg \varepsilon_{lt}$,频率相对于纯静电的变化
\begin{equation}
\frac{\Delta\omega}{\omega}
=\frac{\varepsilon_{lt} \varepsilon_{lt}}
{\ab(\varepsilon_{tt} - k^2 c^2 / \omega^2) \pdv{\varepsilon_{ll}}!{\omega}}
\end{equation}

准电磁
\begin{equation}
\frac{\Delta\omega}{\omega}
=\frac{\varepsilon_{lt} \varepsilon_{lt}}
{\varepsilon_{ll} \pdv{\ab(\varepsilon_{tt} - k^2 c^2 / \omega^2)}!{\omega}}
\end{equation}

\section{平行磁场方向传播的电磁波}

\subsection{左右旋波}

\begin{equation}
\omega^2 = k_\parallel^2 c^2 + 2 \pi \omega \sum_\alpha \omega_{p,\alpha}^2
\iint \frac{v_\perp^3 \kappa_\alpha}{k_\parallel v_\parallel \pm \omega_{c,\alpha} - \omega}
\end{equation}

共振 $\omega - k_\parallel v_\parallel = \pm \omega_{c,\alpha}$

\subsection{哨声波不稳定性}

\section{垂直磁场方向传播的电磁波}

\subsection{电子正常回旋波}

\subsection{离子伯恩斯坦(Bernstein)波}
