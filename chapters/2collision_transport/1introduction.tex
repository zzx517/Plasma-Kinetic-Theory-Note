
\chapter{碰撞和输运概论}

碰撞:粒子相互靠近的瞬间发生的强烈的相互作用。
\begin{description}
    \item[中性粒子] 短程相互作用,力程在原子尺度;
    \item[带点粒子] 长程相互作用,力程为德拜长度。
\end{description}

对于等离子体而言,力程小于德拜长度的相互作用考虑为碰撞,力程大于德拜长度的相互作用考虑为集体运动。

\section{自扩散与集体输运}

\begin{description}
    \item[自扩散] 又称试探粒子近似或弛豫过程。它关注一群试探粒子与等离子体背景粒子间碰撞后,试探粒子受到的影响。它既不考虑试探粒子之间的碰撞,也不考虑背景粒子受到的反作用。
    \item[集体输运] 考虑多群密度相近的粒子间的碰撞,包括同类粒子之间的碰撞。
\end{description}

带碰撞项的动理学方程:
\begin{equation}
\pdv{f_\alpha}{t}
+ \vec{v} \cdot \pdv{f_\alpha}{\vec{r}}
+ \frac{q_\alpha}{m_\alpha} \ab(
    \vec{E} + \vec{v} \times \vec{B}
) \cdot \pdv{f_\alpha}{\vec{v}}
= \ab(\pdv{f_\alpha}{t})_c
\end{equation}

二体碰撞形式:
\begin{equation}
\ab(\pdv{f_\alpha}{t})_c = \sum_\beta C_{\alpha\beta} \ab(f_\alpha, f_\beta)
\end{equation}

对所有碰撞形式:
\begin{subequations}\begin{align}
\int C_{\alpha\alpha} \d^3\vec{v} &= 0 \\
\int m_\alpha\vec{v} C_{\alpha\alpha} \d^3\vec{v} &= 0 \\
\int \frac{1}{2} m_\alpha v^2 C_{\alpha\alpha} \d^3\vec{v} &= 0 \\
\int m_\alpha\vec{v} C_{\alpha\beta} \d^3\vec{v}
+ \int m_\beta\vec{v} C_{\beta\alpha} \d^3\vec{v}&= 0
\end{align}\end{subequations}
对于弹性碰撞
\begin{subequations}\begin{align}
\int C_{\alpha\beta} \d^3\vec{v} &= 0\\
\int \frac{1}{2} m_\alpha v^2 C_{\alpha\beta} \d^3\vec{v}
+ \int \frac{1}{2} m_\beta v^2 C_{\beta\alpha} \d^3\vec{v} &= 0
\end{align}\end{subequations}
定义
\begin{equation}
\vec{R}_{\alpha\beta} = \int m_\alpha\vec{v} C_{\alpha\beta} \d^3\vec{v}
\end{equation}

\subsection{自扩散的描述}

试探粒子:
\begin{equation}
\hat{f}_\alpha(t=0,\vec{v})=\delta(\vec{v}-\vec{u}_\alpha)
\end{equation}

一阶矩,平行于入射方向,动量慢化
\begin{equation}
\odv{p_\parallel}{t} = - \frac{p_\parallel}{\tau_s}
\end{equation}

二阶矩,平行于入射方向,能量衰减
\begin{equation}
\odv{E_\parallel}{t} = - \frac{E_\parallel}{\tau_\parallel}
\end{equation}

二阶矩,垂直于入射方向,能量弥散
\begin{equation}
\odv{E_\perp}{t} = \frac{E_\perp}{\tau_\perp}
\end{equation}

\subsection{集体输运的描述}

流体方程组

摩擦力
\begin{equation}
\vec{F}_{r,\alpha} = \int m_\alpha \vec{v} \ab(\pdv{f_\alpha}{t})_c \d^3\vec{v}
\end{equation}

碰撞交换的热量
\begin{equation}
W_{c,\alpha} = \int \frac{1}{2} m_\alpha \vec{w}_\alpha^2 \left( \pdv{f_\alpha}{t} \right)_c \, \d^3 \vec{v}
\end{equation}

\paragraph{扩散过程}
通量流
\begin{equation}
\vec{\Gamma}_\alpha = - D_\alpha \grad{n_\alpha}
\end{equation}
扩散系数

\paragraph{粘滞过程}
动量流
\begin{equation}
\vec{\Pi}_\alpha = ...
\end{equation}
剪切粘滞系数、体粘滞系数

\paragraph{热传导过程}
热流
\begin{equation}
\vec{q}_\alpha = - \kappa_\alpha \grad{T_\alpha}
\end{equation}
热传导系数
