
\chapter{输运}

\section{磁化等离子体中的经典输运}

\subsection{弛豫时间近似法}

\begin{equation}
\pdv{f_1}{t} + \vec{v} \cdot \pdv{f_0}{\vec{r}}
+ \frac{\vec{F}}{m} \cdot \pdv{f_0}{\vec{v}}
= - \nu f_1
\end{equation}

若考虑定态问题 $\pdv{f_1}!{t}=0$,则
\begin{equation}
f_1 = - \frac{1}{\nu} \ab(\vec{v} \cdot \pdv{f_0}{\vec{r}}
+ \frac{\vec{F}}{m} \cdot \pdv{f_0}{\vec{v}})
\end{equation}

\subsubsection{粒子流与扩散系数}

假定:
\begin{enumerate}
    \item $\grad{n} \neq 0$
    \item $\grad{T} = 0$
    \item $\vec{u} = \vec{0}$
    \item $\vec{F} = \vec{0}$
\end{enumerate}
则
\begin{equation}\begin{aligned}
f_1 &= - \frac{1}{\nu} \vec{v} \cdot \pdv{f_0}{\vec{r}} \\
&=- \frac{\hat{f}_M}{\nu} \vec{v} \cdot \grad{n}
\end{aligned}\end{equation}

粒子流
\begin{equation}\begin{aligned}
\vec{\Gamma} &= \int \vec{v} f_1 \d^3\vec{v} \\
&= - \frac{\grad{n}}{\nu} \cdot \int \hat{f}_M \vec{v}\vec{v} \d^3\vec{v} \\
&= - \frac{\grad{n}}{\nu} \cdot \ab<\vec{v}\vec{v}> \\
&= - \frac{T}{\nu m} \grad{n}
\end{aligned}\end{equation}
扩散系数
\begin{equation}
D = \frac{T}{\nu m}
\end{equation}

\subsubsection{电流、电导率与迁移率}

假定:
\begin{enumerate}
    \item $\grad{n} = 0$
    \item $\grad{T} = 0$
    \item $\vec{u} = \vec{0}$
    \item $\vec{F} = q\vec{E} \neq 0$
\end{enumerate}
则
\begin{equation}
f_1 = - \frac{q}{\nu m} \vec{E} \cdot \pdv{f_0}{\vec{v}}
\end{equation}

电流
\begin{equation}\begin{aligned}
\vec{J} &= q \int \vec{v} f_1 \d^3\vec{v} \\
&= - \frac{q^2}{\nu m} \vec{E} \cdot
\int \pdv{f_0}{\vec{v}} \vec{v} \d^3\vec{v} \\
&= \frac{q^2}{\nu m} \vec{E} \cdot
\int \pdv{\vec{v}}{\vec{v}} f_0 \d^3\vec{v} \\
&= \frac{q^2 n}{\nu m} \vec{E}
\end{aligned}\end{equation}
电导率
\begin{equation}
\sigma = \frac{q^2 n}{\nu m}
\end{equation}

电流引起的粒子流
\begin{equation}
\vec{\Gamma} = \frac{1}{q} \vec{J}
= \frac{q n}{\nu m} \vec{E} = \mu n \vec{E}
\end{equation}
迁移率
\begin{equation}
\mu = \frac{\sigma}{q n} = \frac{q}{\nu m}
\end{equation}

\subsubsection{粘滞张量与粘滞系数}

假定:
\begin{enumerate}
    \item $\grad{n} = 0$
    \item $\grad{T} = 0$
    \item $\vec{u} \neq \vec{0}$、$\dive{\vec{u}} \neq 0$
    \item $\vec{F} = \vec{0}$
\end{enumerate}
则
\begin{equation}
f_1 = - \frac{1}{\nu} \vec{v} \cdot \pdv{f_0}{\vec{r}}
\end{equation}

粘滞张量
\begin{equation}\begin{aligned}
\vec{\Pi} &= \int m \vec{w}\vec{w} f_1 \d^3\vec{v} \\
&= - \frac{m}{\nu} \int \vec{w}\vec{w}
    \vec{v} \cdot \pdv{f_0}{\vec{r}} \d^3\vec{v} \\
&= - \frac{m}{\nu} \int \ab(\vec{v}-\vec{u})\ab(\vec{v}-\vec{u})
    \vec{v} \cdot \pdv{f_0}{\vec{r}} \d^3\vec{v} \\
&= - \frac{m}{\nu} \dive{\int f_0 \vec{v}\vec{w}\vec{w} \d^3\vec{v}}
+ \frac{m}{\nu} \int f_0 \vec{v} \cdot
    \grad{\ab[\ab(\vec{v}-\vec{u})\ab(\vec{v}-\vec{u})]}
 \d^3\vec{v} \\
&= - \frac{m}{\nu} \dive{\int f_0 \vec{v}\vec{w}\vec{w} \d^3\vec{v}}
- \frac{m}{\nu} \int f_0 \ab[
    \ab(\vec{v} \cdot \grad{\vec{u}}) \vec{w}
    + \vec{w} \ab(\vec{v} \cdot \grad{\vec{u}})
] \d^3\vec{v} \\
&= - \frac{m}{\nu} \dive{\ab(\vec{u}\int f_0 \vec{w}\vec{w} \d^3\vec{v})}
- \frac{m}{\nu} \int f_0 \ab[
    \ab(\vec{w} \cdot \grad{\vec{u}}) \vec{w}
    + \vec{w} \ab(\vec{w} \cdot \grad{\vec{u}})
] \d^3\vec{v} \\
&= - \frac{n T}{\nu} \ab[ \ab(\dive{\vec{u}}) \vec{I}
    + \grad{\vec{u}} + \grad{\vec{u}}^\T]
\end{aligned}\end{equation}

粘滞系数
\begin{equation}
\varsigma = \frac{n T}{\nu} = m n D
\end{equation}

\subsubsection{热流与热传导系数}

假定:
\begin{enumerate}
    \item $\grad{n} \neq 0$
    \item $\grad{T} \neq 0$
    \item $\grad{n T} = 0$
    \item $\vec{u} = \vec{0}$
    \item $\vec{F} = \vec{0}$
\end{enumerate}
则
\begin{equation}
f_1 = - \frac{1}{\nu} \vec{v} \cdot \pdv{f_0}{\vec{r}}
\end{equation}

热流
\begin{equation}\begin{aligned}
\vec{q} &= \int \frac{1}{2} m \vec{w}^2 \vec{w} f_1 \d^3\vec{v} \\
&= - \frac{m}{2 \nu} \int \vec{w}^2 \vec{w}
    \vec{v} \cdot \pdv{f_0}{\vec{r}} \d^3\vec{v} \\
&= - \frac{m}{2 \nu} \dive{\ab(n \int \vec{v}^2 \vec{v}
    \vec{v} \hat{f}_M \d^3\vec{v})} \\
&= - \frac{m}{6 \nu} \grad{\ab(n \int v^4 \hat{f}_m \d^3\vec{v})} \\
&= - \frac{5 n T}{2 \nu m} \grad{T}
\end{aligned}\end{equation}
热传导系数
\begin{equation}
\kappa = \frac{5 n T}{2 \nu m}
\end{equation}

\subsection{垂直磁场方向的输运}

\subsubsection{垂直磁场的扩散系数}

除 $\vec{B} \neq 0$ 外,其他假设与之前相同。则
\begin{equation}
\nu f_1 + \vec{v} \cdot \pdv{f_0}{\vec{r}}
+ \frac{q}{m} \ab(\vec{v} \times \vec{B}) \cdot \pdv{f_1}{\vec{v}}
= 0
\end{equation}
其中,若 $f_0$ 各向同性,则
\begin{equation}
\ab(\vec{v} \times \vec{B}) \cdot \pdv{f_0}{\vec{v}} = 0
\end{equation}

\begin{subequations}\begin{align}
D_\perp &= \frac{1}{1 + \omega_c^2 / \nu^2} D_\parallel \\
D_H &= \frac{\omega_c / \nu}{1 + \omega_c^2 / \nu^2} D_\parallel \\
\sigma_\perp &= \frac{1}{1 + \omega_c^2 / \nu^2} \sigma_\parallel \\
\sigma_H &= \frac{\omega_c / \nu}{1 + \omega_c^2 / \nu^2} \sigma_\parallel \\
\kappa_\perp &= \frac{1}{1 + \omega_c^2 / \nu^2} \kappa_\parallel \\
\kappa_H &= \frac{\omega_c / \nu}{1 + \omega_c^2 / \nu^2} \kappa_\parallel
\end{align}\end{subequations}
