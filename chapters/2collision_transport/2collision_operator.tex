
\chapter{碰撞算子}

\section{BGK 碰撞项}

\begin{equation}
\ab(\pdv{f}{t})_c = - \frac{f - f_0}{\tau_c} = - \nu_c \ab(f - f_0)
\end{equation}
弛豫时间近似,适用于偏离平衡态不远的情况。

\section{玻尔兹曼(Boltzmann)碰撞项}

基本假定:
\begin{enumerate}
    \item 两体碰撞;
    \item 碰撞前后自由运动;
    \item 分子混沌,碰撞几率只与 $f_1 f_2$ 有关,与关联函数 $P_{12}(f_1,f_2)$ 无关。
\end{enumerate}

\begin{equation}
\ab(\pdv{f}{t})_c = \iint \ab[
    f_1(\vector{v}_1') f_2(\vector{v}_2')
    - f_1(\vector{v}_1) f_2(\vector{v}_2)
] \ab|\vector{v}_1 - \vector{v}_2| \odv{\sigma}{\Omega}
\d\Omega \d^3\vector{v}_2
\end{equation}

\section{Fokker-Planck 碰撞项}

基本假定:
\begin{enumerate}
    \item 马尔科夫过程;
    \item 小角散射,$\Delta\vector{v}\ll\vector{v}$。
\end{enumerate}

\begin{equation}
\ab(\pdv{f}{t})_c =
- \pdv{}{\vector{v}} \cdot \ab(f \ab<\Delta\vector{v}>)
+ \frac{1}{2} \pdv[2]{}{\vector{v}} \cdot
\ab(f \ab<\Delta\vector{v} \Delta\vector{v}>)
\end{equation}
其中
\begin{subequations}\begin{align}
\ab<\Delta\vector{v}> &= \frac{1}{\Delta t}
\int \Phi(\vector{v}, \Delta\vector{v})
\Delta\vector{v} \d\ab(\Delta\vector{v}) \\
\ab<\Delta\vector{v}\Delta\vector{v}> &= \frac{1}{\Delta t}
\int \Phi(\vector{v}, \Delta\vector{v})
\Delta\vector{v}\Delta\vector{v} \d\ab(\Delta\vector{v})
\end{align}\end{subequations}
分别为动理学摩擦系数、动理学扩散系数。

\section{Rosenbluth 势}

基本假定:
\begin{enumerate}
    \item Fokker-Planck 碰撞项;
    \item 两体碰撞;
    \item 库伦碰撞。
\end{enumerate}

\begin{subequations}\begin{align}
\ab<\Delta\vector{v}> &= \Gamma_1 \pdv{H}{\vector{v}_1} \\
\ab<\Delta\vector{v}\Delta\vector{v}> &=
    \Gamma_1 \pdv{G}{\vector{v}_1,\vector{v}_1} \\
\Gamma_1 = \frac{q_1^4}{\varepsilon_0 m_1^2} \ln\Lambda \\
H(\vector{v}_1) &= \frac{q_2^2}{q_1^2} \frac{m_1 + m_2}{m_2}
\int \frac{f_2(\vector{v}_2)}{\ab|\vector{v}_1 - \vector{v}_2|} \d^2\vector{v}_2 \\
G(\vector{v}_1) &= \frac{q_2^2}{q_1^2}
\int \ab|\vector{v}_1 - \vector{v}_2| f_2(\vector{v}_2) \d^2\vector{v}_2 \\
\end{align}\end{subequations}

\section{朗道(Landau)碰撞项}

基本假定:
\begin{enumerate}
    \item 玻尔兹曼(Boltzmann)碰撞项;
    \item 库伦碰撞;
    \item 小角散射,$\Delta\vector{v}\ll\vector{v}$。
\end{enumerate}

\begin{equation}
\ab(\pdv{f_\alpha}{t})_c =
\frac{\Gamma_\alpha}{2} \sum_\beta \frac{q_\beta^2}{q_\alpha^2} m_\alpha
\pdv{}{\vector{v}_\alpha} \cdot
\int \pdv{\ab|\vector{v}_\alpha-\vector{v}_\beta|}{\vector{v}_\alpha,\vector{v}_\alpha}
\cdot \ab(
    \frac{f_\beta}{m_\alpha} \pdv{f_\alpha}{\vector{v}_\alpha}
    - \frac{f_\alpha}{m_\beta} \pdv{f_\beta}{\vector{v}_\beta}
) \d^3\vector{v}_\beta
\end{equation}

\section{弹性碰撞的守恒性质}

\section{试探粒子的各种碰撞频率}
